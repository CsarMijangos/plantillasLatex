\documentclass[12pt]{article}
\usepackage{amsmath}
\usepackage{amssymb}
\usepackage{amsbsy,amscd,amsfonts,amsthm,amsopn,amstext,amsxtra,euscript,hyperref,latexsym,mathrsfs}

\begin{document}

\def\A{\mathbb{A}}
\def\B{\mathbf{B}}
\def \C{\mathbb{C}}
\def \F{\mathbb{F}}
\def \K{\mathbb{K}}

\def \Z{\mathbb{Z}}
\def \P{\mathbb{P}}
\def \R{\mathbb{R}}
\def \Q{\mathbb{Q}}
\def \N{\mathbb{N}}
\def \Z{\mathbb{Z}}

\def\B{\mathcal B}
\def\e{\varepsilon}

\def\cA{{\mathcal A}}
\def\cB{{\mathcal B}}
\def\cC{{\mathcal C}}
\def\cD{{\mathcal D}}
\def\cE{{\mathcal E}}
\def\cF{{\mathcal F}}
\def\cG{{\mathcal G}}
\def\cH{{\mathcal H}}
\def\cI{{\mathcal I}}
\def\cJ{{\mathcal J}}
\def\cK{{\mathcal K}}
\def\cL{{\mathcal L}}
\def\cM{{\mathcal M}}
\def\cN{{\mathcal N}}
\def\cO{{\mathcal O}}
\def\cP{{\mathcal P}}
\def\cQ{{\mathcal Q}}
\def\cR{{\mathcal R}}
\def\cS{{\mathcal S}}
\def\cT{{\mathcal T}}
\def\cU{{\mathcal U}}
\def\cV{{\mathcal V}}
\def\cW{{\mathcal W}}
\def\cX{{\mathcal X}}
\def\cY{{\mathcal Y}}
\def\cZ{{\mathcal Z}}

\def\f{\frac{|\A||B|}{|G|}}
\def\AB{|\A\cap B|}
\def \Fq{\F_q}
\def \Fqn{\F_{q^n}}

\def\({\left(}
\def\){\right)}
\def\fl#1{\left\lfloor#1\right\rfloor}
\def\rf#1{\left\lceil#1\right\rceil}
\def\Res{{\mathrm{Res}}}

\newcommand{\comm}[1]{\marginpar{
\vskip-\baselineskip \raggedright\footnotesize
\itshape\hrule\smallskip#1\par\smallskip\hrule}}

\newtheorem{lem}{Lemma}
\newtheorem{lemma}[lem]{Lemma}
\newtheorem{prop}{Proposition}
\newtheorem{proposition}[prop]{Proposition }
\newtheorem{thm}{Theorem}
\newtheorem{theorem}[thm]{Theorem}
\newtheorem{cor}{Corollary}
\newtheorem{corollary}[cor]{Corollary}
\newtheorem{prob}{Problem}
\newtheorem{problem}[prob]{Problem}
\newtheorem{ques}{Question}
\newtheorem{question}[ques]{Question}
\newtheorem{rem}{Remark}

\title{A very impressive title}

\author{ {\sc C. F. Gauss}, {\sc L. Euler} and {\sc T. Kaczynski} }

\date{}

\maketitle

\begin{abstract}
In this paper we solve the final problem of mathematics. That is to say, that 
$$
1= 0.
$$
We manage to achieve this breakthrough by a new method that we call "the artisanal method".
\end{abstract}

\paragraph*{2010 Mathematics Subject Classification:}

\paragraph*{Keywords:} {a very important word, another word} 


\section{Introduction}
In this section we ``move our hands'' to distract your attention and make you believe that our work is at the same level of that of the papers that we are going to refer to.

\section{Statement of results}

\begin{theorem}\label{Theo-1}
Let {\tt a list of interesting and not at all ad-hoc and obscure hypothesis} be, then the next {\tt almost negligible improvement on a very interesting problem (very interesting for 5 people in the world)} follows.
\end{theorem}


\begin{theorem}\label{Theo-2}
Another astonishing result maybe a Fields medal worthy.
\end{theorem}

We apply these theorems to obtain new improvements on the following problem ...

As a consequence of Theorem \ref{Theo-1} we derive the following result.


\begin{corollary}\label{Corol-to-Theo-1}
A tiny but not at all negligible improvement to some not at all obscure almost secret problem.
\end{corollary}

As a consequence of
Theorem \ref{Theo-2} we can improve the result in Corollary \ref{Corol-to-Theo-1} over a shorter range of $h$.

\begin{corollary}\label{cinco-variables-corolario}
Another improvement to another result.
\end{corollary}

\section{Lemmas}
 
\begin{lemma}
\label{lemma-1} 
Some result that we are going to need afterwards. 
\end{lemma}

\medskip

\begin{lemma}
\label{lemma-2} 
Un Lema m\'as.
\end{lemma}

We recall the following consequence of \cite[Theorem 4.4]{Mig}.

\begin{lemma}
\label{lemma-3}
And another.
\end{lemma}

\begin{lemma}
\label{lemma-4}
And another.
\end{lemma}

A particular case of Lemma \ref{lemma-1} and Lemma \ref{lemma-3} is the following well-known
result.

\begin{lemma}
\label{lemma-5}
And another
\end{lemma}

\medskip

\begin{lemma}
\label{lemma-6}
This is the last one.
\end{lemma}

\begin{cor} \label{cor-to-lemma-6}
A corollary.
\end{cor}

\section{Proof of Theorem \ref{Theo-1}}
Here goes our impressive argument that gracefully connects the previous lemmas to deduce our breakthrough. 


\section{Proof of Corollary \ref{Corol-to-Theo-1} }
Proof of corollary.

\section{Proof of Theorem \ref{Theo-2}}
Proof theorem \ref{Theo-2}



\pagebreak
\begin{thebibliography}{99}

\bibitem{BHW} Very famous author, another famous author, 
`A good paper', {\it Great Journal \/},
{\bf 9} (1993), 165--175.

\bibitem{AAA} Very famous author, another famous author, 
`An obscure paper', {\it Somalian Journal of Mathematics \/},
{\bf 5} (2000), 185--195.

\end{thebibliography}

\end{document}
