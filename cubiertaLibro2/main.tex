%\documentclass[a4paper]{article}
\documentclass[a4paper,openright,12pt,spanish]{book}
\usepackage[spanish,es-noshorthands,es-nosectiondot]{babel}
\selectlanguage{spanish}
\usepackage[margin=0in]{geometry}
\usepackage{tikz}
\usepackage{amsmath}
\usepackage{amssymb}
\usepackage{xcolor}
\usepackage[T1]{fontenc}
\usepackage{lmodern}

\usetikzlibrary{calc,decorations.pathmorphing,patterns,positioning}

\begin{document}
\pagestyle{empty}

\begin{tikzpicture}[remember picture,overlay]
    % Definir colores
    \definecolor{deepblue}{RGB}{20,50,120}
    \definecolor{lightblue}{RGB}{78,188,255} % azul pastel: {129,207,244} azul 2: {78,188,255}
    \definecolor{accent}{RGB}{220,100,50}
    \definecolor{gold}{RGB}{255,215,0}
    \definecolor{amarillo}{RGB}{246,165,0}
    \definecolor{fondo-amarillo}{RGB}{246,165,0} % amarillo fuerte
    \definecolor{fondo-amarillo-2}{RGB}{241,167,63} % amarillo 2 
    \definecolor{fondo-azul-1}{RGB}{100,207,244} % azul pastel     
    \definecolor{fondo-azul-2}{RGB}{78,188,255} % azul cielo fuerte
    \definecolor{fondo-azul-3}{RGB}{56,181,244} % azul cielo fuerte 2
    \definecolor{fondo-azul-4}{RGB}{30,151,186} % azul cielo fuerte 3
    \definecolor{fondo-azul-5}{RGB}{19,182,234} % azul cielo fuerte 4
    \definecolor{fondo-verde-1}{RGB}{186,208,18} % verde pastel
    \definecolor{fondo-verde-2}{RGB}{166,210,195} % verde pastel 2  
    \definecolor{fondo-verde-3}{RGB}{36,147,82} % verde fuerte 1 
    \definecolor{fondo-melon-1}{RGB}{223,100,96} % melón 1  
    \definecolor{fondo-melon-2}{RGB}{223,127,119} % melón 2   
    \definecolor{fondo-rosa-1}{RGB}{242,156,234} % rosa 1
    \definecolor{fondo-rosa-2}{RGB}{213,149,225} % rosa 2
    \definecolor{fondo-naranja-1}{RGB}{235,104,65} % naranja 1
    \definecolor{fondo-naranja-2}{RGB}{204,42,54} % naranja 2

    % Gradiente de color para el fondo
    %\fill[left color=deepblue, right color=lightblue] 
    %    (current page.north west) rectangle (current page.south east);
    
    % Color solido en el fondo
    \fill[color=fondo-azul-5] 
        (current page.north west) rectangle (current page.south east);

    % Elemento decorativo: segmento diagonal
    %\foreach \i in {0,1,2,3,4} {
    %    \draw[color=accent, opacity=0.3, line width=2pt] 
    %        ([xshift=\i*2cm, yshift=-\i*1.5cm]current page.north west) 
    %        -- ++(4cm,-3cm);
    %}
    
    % Elemento decorativo: circulos
    \foreach \i in {1,2,3} {
        \draw[color=gold, opacity=0.4, line width=1.5pt] 
            ([xshift=15cm+\i*1cm, yshift=-8cm-\i*2cm]current page.north west) 
            circle (0.8cm);
    }
    
    % Título del libro
    \node[black, align=center, font=\fontsize{40}{48}\selectfont\bfseries] 
        at ([yshift=-4cm]current page.north) {
        \textsc{Título Principal}\\
        \vspace{0.5cm}
        {\fontsize{24}{28}\selectfont\textit{Subtítulo}}
    };
    
    % Autor
    \node[white, align=center, font=\fontsize{18}{22}\selectfont] 
        at ([yshift=-22cm]current page.north) {
        \textsc{Nombre del Autor}
    };
    
    % Ecuaciones matemáticas 
    
    % Identidad de Euler (arriba-derecha)
    \node[white, align=center, font=\fontsize{16}{20}\selectfont] 
        at ([xshift=4cm, yshift=-8cm]current page.north) {
        $\displaystyle e^{i\pi} + 1 = 0$
    };
    
    % Teorema Fundamental del Cálculo (izquierda)
    \node[white, align=center, font=\fontsize{14}{18}\selectfont, rotate=15] 
        at ([xshift=-6cm, yshift=-12cm]current page.north) {
        $\displaystyle \int_a^b f'(x)\,dx = f(b) - f(a)$
    };
    
    % Zeta de Riemann (derecha-abajo)
    \node[white, align=center, font=\fontsize{14}{18}\selectfont] 
        at ([xshift=5cm, yshift=-16cm]current page.north) {
        $\displaystyle \zeta(s) = \sum_{n=0}^{\infty}\frac{1}{n^s}$
    };
    
    % Serie de Taylor (abajo-izquierda)
    \node[white, align=center, font=\fontsize{12}{16}\selectfont, rotate=-10] 
        at ([xshift=-5cm, yshift=-18cm]current page.north) {
        $\displaystyle f(x) = \sum_{n=0}^{\infty} \frac{f^{(n)}(a)}{n!}(x-a)^n$
    };
    
    % Integral Gaussiana (centro)
    \node[white, align=center, font=\fontsize{14}{18}\selectfont] 
        at ([xshift=0cm, yshift=-14cm]current page.north) {
        $\displaystyle \int_{-\infty}^{\infty} e^{-x^2} dx = \sqrt{\pi}$
    };
    
    % Teorema de Pitágoras (arriba-izquierda)
    \node[white, align=center, font=\fontsize{14}{18}\selectfont, rotate=25] 
        at ([xshift=-4cm, yshift=-10cm]current page.north) {
        $\displaystyle a^2 + b^2 = c^2$
    };
    
    % Regla de la cadena (derecha)
    \node[white, align=center, font=\fontsize{12}{16}\selectfont, rotate=-20] 
        at ([xshift=6cm, yshift=-12cm]current page.north) {
        $\displaystyle \frac{d}{dx}[f(g(x))] = f'(g(x)) \cdot g'(x)$
    };
    
    % Borde decorativo
    \draw[white, line width=3pt, opacity=0.8] 
        ([xshift=1cm, yshift=-1cm]current page.north west) 
        rectangle 
        ([xshift=-1cm, yshift=1cm]current page.south east);
        
    % Borde interior
    \draw[color=gold, line width=1pt, opacity=0.6] 
        ([xshift=1.2cm, yshift=-1.2cm]current page.north west) 
        rectangle 
        ([xshift=-1.2cm, yshift=1.2cm]current page.south east);
    
\end{tikzpicture}

\newpage

% Cubierta trasera
\begin{tikzpicture}[remember picture,overlay]
    % Define colors
    \definecolor{deepblue}{RGB}{20,50,120}
    \definecolor{lightblue}{RGB}{100,150,220}
    \definecolor{accent}{RGB}{220,100,50}
    \definecolor{gold}{RGB}{255,215,0}
    % Background gradient (reversed)
    \fill[left color=lightblue, right color=deepblue] 
        (current page.north west) rectangle (current page.south east);
    
    % Elementos decorativos
    \foreach \i in {0,1,2,3} {
        \draw[color=accent, opacity=0.2, line width=1.5pt] 
            ([xshift=\i*3cm, yshift=-\i*2cm]current page.north east) 
            -- ++(-5cm,-4cm);
    }
    
    % Contenido
    \node[white, align=justify, font=\fontsize{14}{18}\selectfont, text width=12cm] 
        at (current page.center) {
        \textbf{Acerca de este Libro}\\[0.5cm]
        
        Esta obra contiene una exploración completa de las ideas y conceptos sobre el tema X,
        tendiendo un puente entre los fundamentos teóricos y las aplicaciones prácticas. 
        
        \vspace{0.5cm}
        
        Las características clave incluyen:
        \begin{itemize}
            \item Rigor en el manejo de las ideas
            \item Explicaciones extensivas de los conceptos más complicados
            \item Aplicaciones en el mundo real y casos de estudio
            \item Contexto histórico y nuevas ideas
        \end{itemize}
        
        \vspace{0.5cm}
        
        Perfecto para el estudiante avanzado y para entusiastas y profesionales que buscan
        profundizar su entendimiento en la materia.
        
        \vspace{1cm}
        
        \textbf{Acerca del Autor}\\[0.3cm]
        
        [Breve Semblanza del Autor]
    };
    
    % Información de la editorial (abajo)
    \node[white, align=center, font=\fontsize{12}{16}\selectfont] 
        at ([yshift=2cm]current page.south) {
        \textsc{Publicación Independiente}\\
        \textit{Del pueblo para el pueblo. El conocimiento nos hará libres.}
    };
    
    % ISBN
    \node[white, align=left, font=\fontsize{10}{12}\selectfont] 
        at ([xshift=2cm, yshift=1cm]current page.south west) {
        ISBN: 271-8-281828-45-9
    };
    
    % Borde decorativo
    \draw[white, line width=3pt, opacity=0.8] 
        ([xshift=1cm, yshift=-1cm]current page.north west) 
        rectangle 
        ([xshift=-1cm, yshift=1cm]current page.south east);
        
    % MArca de agua con un símbolo matemático
    \node[white, opacity=0.1, font=\fontsize{120}{140}\selectfont] 
        at ([xshift=2cm, yshift=2cm]current page.center) {
        $ \pi$
    };
    
\end{tikzpicture}

\end{document}