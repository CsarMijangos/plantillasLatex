\documentclass[14pt,compress,usenames,dvipsnames,aspectratio=169]{beamer}
%\usetheme{Singapore}
\useoutertheme{shadow}
\usetheme{CambridgeUS}
\definecolor{mygreen}{RGB}{150, 255, 210}%186}
\definecolor{leftblue}{RGB}{230,255,255}
\definecolor{rightblue}{RGB}{111,195,223}
\definecolor{lefttron}{RGB}{19,44,65}
\definecolor{myblack}{RGB}{27,27,27}
\definecolor{mypurple}{RGB}{205,87,255}

\usecolortheme{owl}

% \setbeamercolor{section in head/foot}{fg = white,bg=black}
\setbeamercolor{title}{fg=mygreen,bg=black}
\setbeamercolor{titlelike}{fg=yellow,bg=black}
\setbeamercolor{item}{fg=mygreen}
\setbeamercolor{block title}{fg=white,bg=myblack!200}
\setbeamercolor{block body}{bg=normal text.bg!80}
\setbeamertemplate{blocks}[rounded][shadow=true]
\setbeamertemplate{headline}{}
\setbeamertemplate{footline}[frame number]
\setbeamercolor{normal text}{fg=white,bg=myblack}%!89.9}

%Gradient
\setbeamercolor{frametitle}{fg=orange,bg=black}
\setbeamercolor{frametitle right}{fg=white,bg=gray}

\usepackage[utf8]{inputenc}
\usepackage{amsmath}
\usepackage{amsfonts}
\usepackage{amssymb}
\usepackage{graphicx}
\usepackage{shadowtext}
\usepackage{multicol}
\usepackage[makeroom]{cancel}

%\graphicspath{{./figures/},
%}

%\AtBeginSection{\frame{\sectionpage}}

\usepackage{natbib}
\usepackage{float}
\usepackage{subcaption}
\usepackage{xcolor}
\usepackage{natbib}
\usepackage{bibentry}
\usepackage{animate}
\usepackage{varwidth}
\usepackage{appendixnumberbeamer}

\usepackage{tikz}
\usetikzlibrary{shapes,arrows}

\title{\textbf{Teoría de Números}}
%\author{Fname Lname}

%\institute[]{%
%Institute Name
%}
\date{}

\usefonttheme{professionalfonts}

\usepackage{mydefs}

\setbeamercovered{transparent} 
\setbeamertemplate{navigation symbols}{} 
\titlegraphic{
\begin{center}
\vspace*{-30pt}
\includegraphics[height=0.5\tH]{figures/gauss_ia.jpg}

\vspace*{10pt}
%\includegraphics[height=0.03\tH]{figures/mail_logo.png}\hspace*{2pt}
%{\scriptsize \href{mailto:abarik@jhu.edu}{email@example.com}}\hspace*{20pt}
%\includegraphics[height=0.03\tH]{figures/twit.png}\hspace*{2pt}
%{\scriptsize \href{https://twitter.com/MHDwizard}{@TwitterHandle}}
\end{center}
}
%\institute[JHU]{Planetary Interiors}



\begin{document}

\setbeamercovered{invisible}

\begin{frame}[plain]
\titlepage
\end{frame}

\begin{frame}
  \frametitle{Contenido}
  \begin{itemize}
    \item Divisibilidad 
    \item Máximo común divisor y mínimo común múltiplo
    \item Algoritmo de Euclides
    \item Números primos y el Teorema fundamental de la aritmética
  \end{itemize}
\end{frame}

\begin{frame}{Divisibilidad}
    \begin{notation}
        $\mathbb{Z}$ denota los enteros y $\mathbb{N}$ denota a los enteros positivos (los naturales).
    \end{notation}
    
    
    \begin{defi}[Divisibilidad]
    Dados $a,b \in \Z$ con $a\neq 0$, decimos que $a$ divide a $b$ (o que $b$ es un múltiplo de $a$), y escribimos $a | b$, si y solo si existe $c \in \Z$ tal que $b=ac$. Si $a$ no divide  a $b$ escribimos $a\nmid b$.
    \end{defi}
\end{frame}

\begin{frame}{Divisibilidad}
    Ejemplos:
    \begin{itemize}
        \item $\forall a\in \Z$ se cumple que
        \begin{enumerate}
        \item $1|a$.
        \item $0\nmid a$.
        \item Si además $a\neq 0$, $a|a$.
        \end{enumerate}
    \end{itemize}
\end{frame}

\begin{frame}{Referencias}
    \bibliographystyle{apalike}
    \nocite{*}
    \bibliography{bib}
\end{frame}

\end{document}