% Carga de paquetes
\usepackage[spanish,es-noshorthands,es-nosectiondot]{babel}
\selectlanguage{spanish}
\usepackage[utf8]{inputenc} % Codificación UTF-8
\usepackage[T1]{fontenc}    % Salida de caracteres
\usepackage{amsmath, amssymb, amsthm} % Matemáticas
\usepackage{amsfonts,latexsym,cancel}
\usepackage{graphicx}       % Imágenes
\usepackage[figurename=Fig.]{caption}

\usepackage[breaklinks=true]{hyperref}       % Hipervínculos
\usepackage{xcolor}         % Colores
\usepackage{geometry}       % Márgenes
%\usepackage[total={6.3in,9in},top=1in,left=1.4in]{geometry}  %Márgenes
\usepackage{fancyhdr}       % Encabezados y pies de página

\usepackage{enumerate}
\usepackage{array}

%\usepackage{appendix}

%%%%%%%%%%%% Indices alfabetico, glosario, simbolos %%%%%%%%%%%
\usepackage{makeidx} % Permite crear índices
\makeindex 
\usepackage[totoc]{idxlayout}
%%%%%%%%%%%%%%%%%%%%%%%%%%%%% GLosarios %%%%%%%%%%%%%%%%%%%%%%%%%%%
\usepackage[nonumberlist,acronym,toc]{glossaries}
\makeglossaries

%%%%%%%%%%%%%%%%% Para agregar a la tabla de contenidos:
%\usepackage[nottoc,notbib]{tocbibind}


%%%%%%%%%%%%%%%%%%%%%%%%% Tabla de contenidos%%%%%%%%%%%%%%%%%%%
\usepackage{tocloft}

%Cambiar título del índice de contenidos:
\renewcommand{\contentsname}{Índice}
% Cambiar profundidad del índice:
\setcounter{tocdepth}{2} % Incluye hasta 0: capítulos; 1: secciones; 2: subsecciones; 3: subsubsecciones; así sucesivamente

% Cambiar el tamaño de los títulos en el índice de contenidos
\renewcommand{\cfttoctitlefont}{\huge\bfseries}
\renewcommand{\cftchapfont}{\bfseries}
\renewcommand{\cftsecfont}{\itshape}
\renewcommand{\cftchappagefont}{\bfseries}
\renewcommand{\cftsecpagefont}{\itshape}


%%%%%%%%%% Índice de resultados importantes %%%%%%%%%%%%%%%%%%%%
\renewcommand{\indexname}{Índice Resultados importantes} %Renombrar el índice alfabético




\usepackage{emptypage}
\usepackage{float}
\usepackage{setspace}
\usepackage{cases}


% Configuración de márgenes
\geometry{margin=1in}

% Estilo de encabezado
\pagestyle{fancy}
\fancyhf{}
\fancyhead[L]{\leftmark}
\fancyhead[R]{\thepage}

% Cargar estilos personalizados
% Cambiar fuentes para teoremas y demostraciones
\usepackage{lmodern}
\usepackage{titlesec}

% Estilo para teoremas
\newtheoremstyle{mytheorem}
  {10pt} % Espacio arriba
  {10pt} % Espacio abajo
  {\itshape} % Fuente del cuerpo
  {} % Sangría
  {\bfseries\color{blue}} % Fuente del título
  {} % Puntuación tras el título
  { } % Espacio tras el título
  {%
    \thmname{#1}~\thmnumber{#2}%
    \if\relax\detokenize{#3}\relax % Comprueba si la nota está vacía
    .\else~(\thmnote{#3}).\fi
  } % Formato del título

% Estilos numerados
\theoremstyle{mytheorem}
\newtheorem{theo}{Teorema}[chapter]
\newtheorem{lemma}[theo]{Lema}
\newtheorem{corol}[theo]{Corolario}
\newtheorem{conje}[theo]{Conjetura}
\newtheorem{ejem}{Ejemplo}[chapter]
\newtheorem{ejer}{Ejercicio}[chapter]

% Estilo para definiciones
\newtheoremstyle{mydefinition}
  {10pt}
  {10pt}
  {\normalfont} % Fuente normal (sin cursiva)
  {}
  {\bfseries\color{teal}}
  {}
  { }
  {%
    \thmname{#1}~\thmnumber{#2}%
    \if\relax\detokenize{#3}\relax % Comprueba si la nota está vacía
    \else~(\thmnote{#3}).\fi
  } % Formato del título

\theoremstyle{mydefinition}
\newtheorem*{defi}{Definición.}

% Estilo para demostraciones
\newtheoremstyle{mydemostration}
  {10pt}
  {10pt}
  {\normalfont} % Fuente normal (sin cursiva)
  {}
  {\bfseries}
  {}
  { }
  {\thmname{#1}~\thmnumber{#2}~\thmnote{(#3)}}

\theoremstyle{mydemostration}
\newtheorem*{dem}{Demostración.}
\renewcommand{\qedsymbol}{$\blacksquare$}
\AtEndEnvironment{dem}{\null\hfill\qedsymbol}

% Estilos no numerados
\theoremstyle{remark}
\newtheorem*{obs}{\textbf{Observación}}
\newtheorem*{afir}{\textbf{Afirmación}}


% Definir comandos personalizados
\newcommand{\R}{\mathbb{R}} % Conjunto de los números reales
\newcommand{\Z}{\mathbb{Z}} % Conjunto de los números enteros
\newcommand{\N}{\mathbb{N}} % Conjunto de los números naturales
\newcommand{\Q}{\mathbb{Q}} % Conjunto de los números racionales
\newcommand{\C}{\mathbb{C}} % Conjunto de los números complejos
\newcommand{\abs}[1]{\left\lvert#1\right\rvert} % Valor absoluto
\newcommand{\norm}[1]{\left\lVert#1\right\rVert} % Norma
\newcommand{\set}[1]{\left\{#1\right\}} % Conjunto
\newcommand{\inner}[2]{\langle#1, #2\rangle} % Producto interno

\newcommand{\tq}{\, : \,} % Conjunto de los números complejos
