\chapter{Pruebas de newcommands, index, glossary, etc}

Este es el capítulo 1 de tu libro de matemáticas.

\section{Ejemplo de Teorema}

\begin{theo}[Pitagoras]
En un triángulo rectángulo, el cuadrado de la hipotenusa es igual a la suma de los cuadrados de los catetos.
\end{theo}

\begin{dem}
Sea un triángulo con lados $a$, $b$ y $c$, donde $c$ es la hipotenusa. Según el Teorema de Pitágoras:
\[
c^2 = a^2 + b^2.
\]
\end{dem}

\begin{afir}
Esta es una afirmación.
\end{afir}

\begin{obs}
    Esta es una observación
\end{obs}

\begin{defi}
Esta es una definición
\end{defi}

\begin{lemma}
    Este es un lema.
\end{lemma}

\begin{corol}
    Este es un corolario valor absoluto: $\abs{x}$, norma: $\norm{x}$, 
    conjunto $$A=\set{a_1,a_2,a_3}$$
\end{corol}

\begin{ejem}
    Este es un ejemplo
\end{ejem}
\section{Ejemplo de índice}
Este es un ejemplo de cómo usar el comando \index{Teorema} para generar índices.
\section{Ejemplo de glosario}
\begin{theo}
Este teorema es importante.\index{Teorema!Importante}
\end{theo}

Un \gls{teorema} es un concepto fundamental.